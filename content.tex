% For easier proof-reading, use the single-column, double-spaced layout:
\documentclass{IWORK2014}
% Final Paper use double-column, normal line spacing. Comment the
% above line and uncomment the following line when you are writing
% Full paper and Final paper!  
%\documentclass[cameraready]{IWORK2014}

\usepackage[hyphenbreaks]{breakurl}
\usepackage{hyperref}
\begin{document}

%=========================================================

\title{Bio-Inspired Wireless Networking}

\author{Juha Viljanen\\
        Aalto University School of Science \\
	\texttt{juha.o.viljanen@aalto.fi}}
\maketitle

%==========================================================

\begin{abstract}
  Abstract will be here...

\vspace{3mm}
\noindent KEYWORDS: bio-inspired, networking

\end{abstract}

%============================================================


\section{Introduction}
\begin{itemize}
	\item Why are bio-inspired approaches interesting?
	\item Where are bio-inspired approaches commonly used?
	\item What commonly used approaches are there?
	\item Where are they used?	
\end{itemize}

\section{Background}
Definitions of commonly used terms here.

\section{Results}
\begin{itemize}
	\item List most commonly used approaches here.
	\item How does their archetype in the biological system work?
	\item How is it applied?
	\item Explain the alogrithm.
\end{itemize}


\section{Conclusion}
\begin{itemize}
	\item Why the explained bio-inspired approaches are used / why they are good.
	\item Some future aspects on possible directions of bio-inspired networking.
\end{itemize}


%============================================================

\bibliography{references}
\end{document}

