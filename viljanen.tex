% For easier proof-reading, use the single-column, double-spaced layout:
\documentclass{IWORK2014}
% Final Paper use double-column, normal line spacing. Comment the
% above line and uncomment the following line when you are writing
% Full paper and Final paper!  
%\documentclass[cameraready]{IWORK2014}

\usepackage[hyphenbreaks]{breakurl}
\usepackage{hyperref}
\usepackage{tabularx}
\begin{document}

%=========================================================

\title{Bio-Inspired Networking}

\author{Juha Viljanen\\
        Aalto University School of Science \\
	\texttt{juha.o.viljanen@aalto.fi}}
\maketitle

%==========================================================

\begin{abstract}
In nature evolution has made sure that only the fittest biological organisms and systems survive. They tend to have simple and elegant solutions to complex problems arising from challenging environmental conditions. Throughout history, man has drawn inspiration from nature's solutions to solve issues in totally different fields. In recent decades these nature and biology inspired approaches have been applied in the field of computer science and networking.

Developing bio-inspired solutions is typically divided into three steps: Finding the analogies between the contexts of a technical problem and the biological system; deepening the understanding on the biological system; simplifying the biological patterns and applying them to the technical solution.

This paper takes a closer look at two bio-inspired networking solutions: the A-ESR algorithm, which is based on Ant Colony Optimization (AOC), and the Physarum Optimization (PO). AOC is drawn from higher level understanding of a biological system: a foraging ant colony. The A-ESR algorithm is used in the Internet infrastructure to optimize its energy consumption. Physarum Optimization draws understanding from a lower abstraction level: the functions of a simple cellular organism. It is used to find the minimum exposure path of a wireless sensor network, i.e. finding the path between two predefined points in a sensor-covered area that has the least sensor-coverage.

This paper looks at similarities and differences between the two approaches and considers some other possible application areas for them.

\vspace{3mm}
\noindent KEYWORDS: bio-inspired, networking, Ant Colony Optimization, A-ESR, Physarum Optimization

\end{abstract}

%============================================================


\section{Introduction}

Nature provides different mechanisms to solve complex problems in pragmatic, efficient and elegant ways. There is a lot that we can learn from Nature and apply to the world of computing and networking. One can find analogies between the different players and functions of a specific system in nature and a technical problem, for example in the networking domain \cite{dressler2010bio}.

Inspiration has historically been drawn from a higher level understanding of biological systems in Nature \cite{kroeker2011biology}, \cite{liu2012physarum}. A representative example of this is the sophisticated way ants passively communicate with each other while foraging. They do not need any centralized controlling unit nor do they ever need to have physical contact with each other. By leaving and following the right pheromone trails specified by a certain set of rules, they are able to coordinate their foraging in an efficient and autonomous way. A similar solution has been adapted and applied to a modern routing protocol \cite{dressler2010bio}.

Today, also solutions from low level biological systems, such as molecular systems \cite{kroeker2011biology} or simple cellular organisms \cite{liu2012physarum}, are adapted to the field of computer science. For example, the neurological development of fruit-flies has inspired a minimalistic and efficient algorithm for self-organizing distributed networks \cite{kroeker2011biology}.

\subsection{What is Bio-inpired networking?}
Bio-inspired approaches are employed in three main areas, namely computing, systems and networking. In the area of computing, bio-inspired approaches are exploited to improve the efficiency of computation algorithms, for example in optimization processes or pattern matching. Regarding the systems area, research is ongoing to design system architectures of massivley distributed, collaborative systems, while the area of networking is already benefiting from efficient and scalable networking solutions and autonomous organizing in a distributed environment. \cite{dressler2010bio}

\subsection{What makes Bio-inspired approaches attractive?} 
Bio-inspired approaches offer qualities attractive to the networking field. They offer feasible solutions for achieving the demanding characteristics of next generation network architectures. The most challenging characteristics are the dynamic nature of mobile and ad-hoc networks and cognitive radio networks, autonomous operation in a infrastructureless network and communication in nano and micro scale networks \cite{dressler2010bio}. In large, heterogenous networks, the most efficient solutions do not usually include a centralized controlling entity, but favor a self-organizing, learning and evolving type of agents traversing and finding optimal routes through the network \cite{dressler2010bio}.

Biological systems often have a small set of simple rules that can be used to create complex behavior \cite{dressler2010bio} for solving challenging problems. These solutions usually possess qualities such as self-organization \cite{kroeker2011biology}, adaptation to changing environmental conditions, fault tolerance, efficient management of scarce resources, collaboration and survival to harsh conditions \cite{dressler2010bio}.

\section{Background}

\subsection{Developing a bio-inspired approach}
Developing a bio-inspired approach for issues in the engineering field is divided into three steps \cite{dressler2010bio}. At the beginning one needs to identify the analogies between the targeted technical environment and the biological system. How do the different players, structures and methods in the biological system correspond to those in the technical environment? Are there similarities? After finding the analogies, the next step is to research the biological system and try to understand it and its functions in a detailed and precise way. The third step is usually to generalize the biological model and to apply it to the chosen technical problem, hence to define the bio-inspired solution.

\section{Results}

This section describes more closely two different bio-inspired approaches: Ant Colony Optimization and Physarum Optimization. Both are optimization algorithms based on the foraging of biological entities. Ant Colony Optimization is a text book example of a bio-inspired approach based on a higher level understanding of a biological system: foraging ants. Physarum Optimization by contrast is based on a lower level understanding on how simple cellular organisms work and the evolution of their behaviour \cite{liu2012physarum}.

\subsection{Ant Colony Optimization}
Ant Colony Optimization (ACO) is a form of bio-inspired solutions derived from the category of Swarm Intelligence \cite{hylsberg2011bioinspired}. One of its applications, the AntNet routing protocol, was introduced already in 1998 \cite{di1998antnet} with the purpose of exploiting ACO in order to improve the routing protocols efficiency in an autonomous fashion. More recently, Y.-M. Kim et al \cite{kim2012ant} proposed a new routing scheme called A-ESR to optimize network elements and hence to make the whole Internet more energy efficient, \cite{kim2011ant}. This study examines the A-ESR algorithm.

\subsubsection{The technical problem and its context: Energy effient routing of Internet traffic}
The infrastructure of the Internet is consuming more and more energy every year. In 2008, studies showed that the Internet consumed about 5.5\% of the worlds total energy. This share was, at that time, expected to grow at a rate of 20-25\% annually \cite{proceedings2008energy}. A substantial portion of Internets power consumption could be reduced by optimizing the network elements, this making them more energy-efficient \cite{andrews2010routing}. Network elements use the same amount of power regardless of whether there is a lot of traffic going through them during peak hours, or whether they are in an idle state during low traffic hours \cite{gupta2007using}.

\subsubsection{The biological system: Foraging ant colonies}
Foraging ants of an ant colony are able to over time find the shortest paths from their nest to food sources in its environment \cite{dorigo1999ant, goss1989self}. They do this using indirect, local communication called stigmergy \cite{grasse1959reconstruction}. While moving between their nest and a food source the ants leave behind a chemical substance called pheromone \cite{ghosh2008aggregation}. A ant following a shorter path can complete it more frequently than one following a longer one. Hence the shorter paths get stronger pheromone trails.

Ants prefer to move to a direction that has the most intense pheromone trace, which usually is the shortest path to a food source. This way the shortest paths quickly get significantly stronger pheromone trails and will attract even more ants. Over time the pheromone fades away. Hence longer routes or paths to food sources that have run out will completely fade away over time. A part of the ants does random searches to be able to find new food sources or new, shorter paths to old food sources \cite{dressler2010bio}.

\subsubsection{Analogies and application: Basic concepts of ACO}
The idea behind classic Ant Colony Optimization algorithm is to do path sampling in a network \cite{di2004ant, di2008theory}. The sampling is done traversing control data packets through the network and recording the performance and delays between each node. The control packages are generally referred to as agents or artificial ants. They will try out different paths between two arbitary points in the network. We distinct two types of agents: forward and backward ants.

Forward ants are created concurrently and independently at different nodes of the network. They have a predefined, but randomly chosen destination that they traverse to starting from the node they were created in. On their way to the destination they will follow a set of rules to find optimal routes. They gather information on all the delays between each jump between nodes.

Once a forward ant reaches its destination a backward ant is created to traverse the same way back to the origin of the forward node. On its way it will update the so called pheromone tables of each node. The pheromone tables are data structures that contain information on how good routes to different destinations traversing through its neighbour nodes are, i.e. the desirability of travesing to a node, i, next to the current node, j, when the packet destination is node j. The forward ants going through a node will choose the target of their next jump based on a stochastic rule that gives a higher probability to neighbour nodes that have stronger pheromone trails, i.e. are suggested by the pheromone table as an optimal path to the agents destination. The stochastic rule also applies a local heuristic function differing between ACO implentations. This way the exploration of new paths as well as load balancing of different routes can be accomplished.

\begin{table}
	\begin{tabularx}{\linewidth}{|X|X|X|}
		\hline & \textbf{Foraging ant colony} & \textbf{Ant Colony Optimization} \\ \hline
		\textbf{Context} & Foraging ant colony & Networking \\ \hline
		\textbf{Problem} & Finding optimal paths to food sources & Finding optimal paths between network elements \\ \hline
		\textbf{Players} & Ants & Agents: forward and backward ants \\ \hline
		\textbf{Targets} & Food sources & Predefined destinations of each forward ant \\ \hline
		\textbf{Communication method} & Trails of chemical substance called pheromone & Numerical values written by backward ants to tables of each network element \\ \hline
	\end{tabularx}
	\caption{Analogies between foraging anto colonies and the Ant Colony Optimization algorithm}
	\label{tbl:analogies_ant}
\end{table}

\subsubsection{How the A-ESR algorithm differs from classic AOC}
% explain what's new in A-ESR

A-ESR stands for self-adaptive energy saving routing \cite{kim2012ant}. The advantages of the A-ESR algorithm are its means to self-adjust against wrong measures and the ability to continuously keep track about the network status.

\subsection{Physarum Optimization}
The Physarum Optimization approach exploits the cellular computation model of the slime mold physarum polycephalum in wireless sensor networks (WSN) \cite{liu2012physarum}. It is used to solve the minimal exposure path problem of WSNs, which relateds to their worst case coverage. The approach is claimed to be simple and its processing highly concurrent. It should also be useful for designing new graph-algorithms, routing protocols and self-organizing network (SON) topologies.

\subsubsection{The technical problem and its context: Wireless sensor networks}
Wireless sensor networks consist of sensor nodes geographically distributed in an area \cite{nazi2013robust}. These sensor nodes collect data from their environment, process it by aggregating and filtering and forwarding it to other recievers in the sensor network. WSNs may be used in contexts of smart healthcare, disaster management, environment monitoring \cite{nazi2013robust} or even military applications \cite{liu2012physarum}.

WSNs can consist of sensors, let us say motion detectors, that aim to detect intruders coming to a restricted area. Generally the coverage of these areas is not complete and by an intruder might find uncovered paths through the restricted area, that would allow him to trespass without getting noticed. L. Liu et al \cite{liu2012physarum} aim to find the path, in such intruder detection scenario, which has the least sensor coverage and hence has the least risk of the intruder being detected by the sensors. This path represents the minimal exposure path. By identifying the minimal exposure paths one can find the weak spots of the covered area and add new sensors to that critical area, where they improve the areas coverage the most.

\subsubsection{The biological system: Physarum Polycephalum}
Physarum polycephalum is a large, single celled amoeba-like organism which belongs to the family of slime molds \cite{liu2012physarum}. Its body has tube-like constructions that it uses to transfer nutrients, signals. Physarum moves by transporting body mass in these tubes. Physarum is known to avoid light. It has the ability to find optimal routes between food sources -- even if there were a maze separating them \cite{nakagaki2000intelligence}. In a dark, non-illuminated area the optimal route is the shortest route between the food sources. In an unevenly lit environment the optimal path is the one with the least risk of being exposed to light.

\textit{TODO: Mention locality and parallelism of the physarums approach}
\subsubsection{Analogies and application: The Physarum Optimization algorithm}
The analogy between Physarum finding the shortest path between food sources in an inhomogenously illuminated field and the trying to find the path having the worst sensor coverage in a wireless sensor network is simple. In the biological system the desired result is that Physarum does not catch light on his way to the food source. In the wireless sensor networks this corresponds to an intruder that does not want to be exposed to sensors while traversing between the selected target points in a sensor-covered area \cite{liu2012physarum}. Using these analogies the minimal exposure path problem of the WSNs can be solved using observations from Physarum's behaviour.

\begin{table}
	\begin{tabularx}{\linewidth}{|X|X|X|}
		\hline & \textbf{Foraging Physarum Polycephalum} & \textbf{Physarum Optimization} \\ \hline
		\textbf{Context} & Foraging slime mold & Wireless Sensor Networks \\ \hline
		\textbf{Problem} & Finding optimal paths to food sources & Finding optimal paths between to predefined points \\ \hline
		\textbf{What defines an optimal path} & A path that has the least amount of light & A path with the least sensor coverage \\ \hline
		\textbf{Players} & Physarum & Intruder \\ \hline
		\textbf{Targets} & Food sources & Predefined destinations \\ \hline
	\end{tabularx}
	\caption{Analogies between foraging Physarum Polycephalum and the Physarum Optimization algorithm}
	\label{tbl:analogies_physarum}
\end{table}

\section{Conclusion}

\subsection{Comparison between ACO and PO}
\textit{TODO: Insert more similarities and differences between the two approaches.}

\begin{tabularx}{\linewidth}{|X|X|X|}
\hline & \textbf{Ant Colony Optimization, A-ESR} & \textbf{Physarum Optimization} \\ \hline
\textbf{View point on biological system} & High level & Low level \\ \hline
\textbf{Application field} & Computer networking, Energy efficient routing & Wireless Sensor Networks, Minimal exposure path \\ \hline
\textbf{Control system} & Locality (no central control necessary). Allows easy parallelity. & Locality (no central control necessary). Allows easy parallelity. \\ \hline
\end{tabularx}

\subsection{Is there a general difference between high and low level inspired approaches?}
\textit{What is the difference? Is either one better? Are low level functions more advanced?}

The comparison between Physarum Optimization and A-ESR indicate that there is no substancial difference between bio-inspired approaches that are inspired by higher or lower level understanding of biolocial systems. They both follow the same rules for defining a bio-inspired solution that were defined by Dressler et al \cite{dressler2010bio}. What is more important is finding the analogies between the biological system and the technical environment. Hence the defining factor of bio-inspired solutions is not the abstraction level of their origin, but rather the type problems they are able to solve.

For example Hylsberg et al \cite{hylsberg2011bioinspired} list problem types and contexts that different biological systems are able to solve and have been successfully applied in. Swarm intelligence has, for instance, been applied to routing in computer networks, optimal node deployment, node localization, and network clustering, whereas the artificial immune system has been applied to misbehavior detection and intrusion detection systems.

\subsection{Why the explained bio-inspired approaches are used and why they are efficient}
\textit{TODO: in the case of physarum, mention locality and parallelism, in ACO mention decentralized control and self-organization.}

\subsection{Future aspects}
As the A-ESR algorithm is used to save energy consumption of network elements routing Internet traffic, it might be worth investigating whether it would be as useful in the field of wireless sensor networks as well. One of the most important limiting factors in WSN is the limited available energy, which sets its challenges to the design of the used routing protocols \cite{hylsberg2011bioinspired}.

Many applications using Ant Colony Optimization in WSNs have been proposed \cite{bennis2013enhanced, zhang2004improvements, camilo2006energy, cai2006aco, sun2008asar, kiri2007self, ghasemaghaei2007ant}. These approaches have different goals ranging from fault tolerance \cite{zhang2004improvements} to reliability and scalability \cite{kiri2007self}. Many of the ACO based algorithms applied in WSN are energy aware \cite{saleem2011swarm}. Also completly othe solutions not applying bio-inspired approaches have been researched for energy effieciency in the WSN field \cite{wightman2008a3}. It would be interesting to see whether the A-ESR approach would be applicable in there and how it would perform in comparison to the other enegry efficient solutions.

\textit{TODO: add reasons pro / con.}

\textit{Could Physarum Optimization be used in networking / routing? Are there possible applications of the minimal exposure path in those fields?}

%============================================================

\bibliography{references}
\end{document}

