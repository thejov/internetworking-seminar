% For easier proof-reading, use the single-column, double-spaced layout:
\documentclass{IWORK2014}
% Final Paper use double-column, normal line spacing. Comment the
% above line and uncomment the following line when you are writing
% Full paper and Final paper!  
%\documentclass[cameraready]{IWORK2014}

\usepackage[hyphenbreaks]{breakurl}
\usepackage{hyperref}
\begin{document}

%=========================================================

\title{Bio-Inspired Networking}

\author{Juha Viljanen\\
        Aalto University School of Science \\
	\texttt{juha.o.viljanen@aalto.fi}}
\maketitle

%==========================================================

\begin{abstract}
  Abstract will be here...

\vspace{3mm}
\noindent KEYWORDS: bio-inspired, networking

\end{abstract}

%============================================================


\section{Introduction}

Nature has a way of solving complex problems in an pragmatic, efficient and elegant way. There is a lot that we can learn and try to adapt to similar issues in totally different fields like computing and networking. One can find analogies between the different players and functions of a specific system in nature and a technical problem in e.g. the networking domain \ref{dressler2010bio}. For example the sophisticated way ants use for foraging offers means for a routing protocol to function more efficiently and without a outside controlling unit \ref{dressler2010bio}.

\subsection{What is Bio-inpired networking?}
Bio-inpired networking is on of the three main areas, where bio-inspired approaches are used. The other areas are Bio-inspired computing and Bio-inspired systems. They are all technical solutions that draw their inspiration from nature. Bio-inspired tries to find effective computation algorithms for e.g. optimization processes or pattern matching. Bio-inspired systems stands for system architectures of massivley distributed, collaborative systems. Bio-inspired networking concentrates on finding solutions for efficient and scalable networking and autonomous organizing in a distributed environment. \ref{dressler2010bio}

\subsection{What makes Bio-inspired approaches attractive?} 
Bio-inspired approaches offer qualities that can be found attractive in the networking field and offer feasable solutions to fulfill the needs of the demanding charasteristics of next generation network architectures. These are the dynamic nature of mobile and ad-hoc networks and cognitive radionetworks, autonomous operation in a infrastructureless network and communicating in nano and micro scale networks \ref{dressler2010bio}. In large, heterogenous networks the most efficient solutions do not usually include a centralized controlling entity, but favor a self-organizing, learning and evolving nature of agents traversing and finding optimal routes through the network \ref{dressler2010bio}.

Biological systems often 




\section{Background}
Definitions of commonly used terms here.

\section{Results}

\begin{itemize}
	\item List most commonly used approaches here.
	\item How does their archetype in the biological system work?
	\item How is it applied?
	\item Explain the alogrithm.
\end{itemize}

\section{Conclusion}
\begin{itemize}
	\item Why the explained bio-inspired approaches are used / why they are good.
	\item Some future aspects on possible directions of bio-inspired networking.
\end{itemize}


%============================================================

\bibliography{references}
\end{document}

