% For easier proof-reading, use the single-column, double-spaced layout:
\documentclass{IWORK2014}
% Final Paper use double-column, normal line spacing. Comment the
% above line and uncomment the following line when you are writing
% Full paper and Final paper!  
%\documentclass[cameraready]{IWORK2014}

\usepackage[hyphenbreaks]{breakurl}
\usepackage{hyperref}
\begin{document}

%=========================================================

\title{Bio-Inspired Networking}

\author{Juha Viljanen\\
        Aalto University School of Science \\
	\texttt{juha.o.viljanen@aalto.fi}}
\maketitle

%==========================================================

\begin{abstract}
\textit{Abstract will be here...}

\vspace{3mm}
\noindent KEYWORDS: bio-inspired, networking

\end{abstract}

%============================================================


\section{Introduction}

Nature provides different mechanisms to solve complex problems in pragmatic, efficient and elegant ways. There is a lot that we can learn from Nature and apply to the world of computing and networking. One can find analogies between the different players and functions of a specific system in nature and a technical problem for example in the networking domain \cite{dressler2010bio}.

Inspiration has classically been drawn from a higher level understanding of biological systems in Nature \cite{kroeker2011biology}, \cite{liu2012physarum}. A representative example of this is the sophisticated way ants passively communicate with each other while foraging. They do not need any centralized controlling unit nor do they ever need to have fysical contact with each other. By leaving and following the right feromone trails specified by a certain set of rules, they are able to to manage their foraging in an efficient and autonomous way. A similar solution has been adapted and applied to a modern routing protocol \cite{dressler2010bio}.

Today also solutions from low level biological systems, such as molecular systems \cite{kroeker2011biology} or simple cellular organisms \cite{liu2012physarum}, are adapted to the field of computer science. For example the neurologic development of fruit-flies has inspired a minimalistic and efficient algorithm for self-organizing distributed networks \cite{kroeker2011biology}.

\subsection{What is Bio-inpired networking?}
Bio-inpired approaches are employed in three main areas. These are computing, systems and networking. They are all technical solutions that draw their inspiration from Nature. In the area of computing bio-inspired approaches are exploited to improve the efficiency of computation algorithms for example in optimization processes or pattern matching. Regarding the systems area, research is ongoing to design system architectures of massivley distributed, collaborative systems. The area of networking is already benefiting from efficient and scalable networking solutions and autonomous organizing in a distributed environment. \cite{dressler2010bio}

\subsection{What makes Bio-inspired approaches attractive?} 
Bio-inspired approaches offer qualities attractive to the networking field. They offer feasable solutions for achieving the demanding charasteristics of next generation network architectures. The most challenging characteristics are the dynamic nature of mobile and ad-hoc networks and cognitive radio networks, autonomous operation in a infrastructureless network and communication in nano and micro scale networks \cite{dressler2010bio}. In large, heterogenous networks, the most efficient solutions do not usually include a centralized controlling entity, but favor a self-organizing, learning and evolving type of agents traversing and finding optimal routes through the network \cite{dressler2010bio}.

Biological systems often have a small set of simple rules that can be used to create complex behavior \cite{dressler2010bio} for solving challenging problems. These solutions usually include qualities such as self-organization \cite{kroeker2011biology}, adaptation to changing environmental conditions, fault tolerance, efficient management of scarce resources, collaboration and survival to harsh conditions \cite{dressler2010bio}.

\section{Background}

\subsection{Developing a bio-inspired approach}
Developing a bioinspired approach for engineering issues is divided into three steps \cite{dressler2010bio}. First one needs to identify the analogies between the target technical environment and the biological system. How do the different players, structures and methods in the biological system correspond to the ones in the technical field? Are there similarities? After finding the analogies the next step is to research the biological system and try to understand it and its functions in a detailled and precise way. The third step is usually to generalize the biological model and to apply it to the chosen technical field, hence to define the bio-inspired solution.

\textit{Definitions of commonly used terms here.}

\section{Results}

In the results we will describe more closely two different bio-inspired approaches: Ant Colony Optimization and Physarum Optimization. Both are optimization algorithms based on foraging of a biological system. Ant Colony Optimization is a text book example of a bio-inspired approach based on a higher level understanding of a biological system: foraging ants. Physarum Optimization  in turn is based on lower level understanding on how simple cellular organisms work and the evolution of its behaviour \cite{liu2012physarum}.

\textit{
\begin{itemize}
	\item Explain the technical problem and its context.
	\item Explain the biological system
	\item Explain the analogies of the two.
	\item Explain the bio-inspired solution for the technical problem.
\end{itemize}
}

\subsection{Ant Colony Optimization}
A classic application of Ant Colony Optimization (ACO) was the AntNet routing protocol that was able to exploit ACO making the protocol function more efficiently and without an outside controlling unit \cite{di1998antnet}. More recently a new routing scheme called A-ESR has been proposed to make network elements and hence the whole Internet to be more energy efficient \cite{kim2012ant}, \cite{kim2011ant}. We will study the more recent application.

\subsubsection{The technical problem and its context: Wireless Sensor Networks}
...

\subsection{Physarum Optimization}
The Physarum Optimization approach exploits the cellular computation model of the slime mold physarum polycephalum in wireless sensor networks (WSN) \cite{liu2012physarum}. It is used to solve the minimal exposure path problem of WSNs, which relateds to their worst case coverage. The approach is claimed to be simple and highly concurrent. It should also be useful for designing new graph-algorithms, routing protocols and Self-Organizing Network (SON) topologies.

\subsubsection{The technical problem and its context: Wireless Sensor Networks}
Wireless Sensor Networks consist of sensor nodes geographically distributed on an area \cite{nazi2013robust}. These sensor nodes collect data from their environment, process it by doing aggregation and filtering and forward it to other recievers in the network they are connected in. For WSNs it is important to be fault tolerant, since network failures may occurr for example due to one of its nodes running out of power. WSNs may be used in contexts of smart healthcare, disaster management, environment monitoring \cite{nazi2013robust} or even military applications \cite{liu2012physarum}.

The case of researched in \cite{liu2012physarum} is one, where a certain area is populated with environment monitoring sensors. The purpose of the sensor network is to detect possible intruders in the monitored area. The goal of the paper is to find the path between two target points going through an area with the least sensor coverage and hence the least risk of being detected by the sensors. This path represents the minimal exposure path. By identifying the least coverage paths one can find the weak spots of the covered area and add the least amount of new sensors that improve its coverage the most.

\subsubsection{The biological system: Physarum Polycephalum}
Physarum polycephalum is a large, single celled amoeba-like organism, which belongs to the group of slime molds \cite{liu2012physarum}. In its body it has tube-like constructions that it uses to transfer nutrients, signals and even its own body mass. Physarum is known to avoid light. It has the ability to find optimal routes between food sources -- even if there would be a maze separating them \cite{nakagaki2000intelligence}. Considering its way of avoiding light, the optimal route means either the shortest route in a non-illuminated place, or in an inhomogenously lit environment the path with the minimum risk to be exposed to light.


locality and parallelism
\subsubsection{Analogies and application: The Physarum Optimization Algorithm}
The analogy between Physarum finding the shortest path between food sources in a inhomogenously illuminated field and the trying to find the path with the worst sensor coverage in a Wireless Sensor Network is simple. Physarum catching light can be considered a corresponing action to an intruder being exposed to a sensor, and the food sources of Physarum resemble the target points between the path in a WSN \cite{liu2012physarum}. Using these analogies the minimal exposure path problem of the WSNs can be solved using observations from Physarum's behaviour.

\section{Conclusion}

\subsection{Comparison between ACO and PO}
\textit{insert table with similarities and differences between the two}
\begin{table}[h]
\begin{tabular}{|l|l|l|}
\hline
                                         & \textbf{Ant Colony Optimization, A-ESR}       & \textbf{Physarum Optimization}              \\ \hline
\textbf{View point on biological system} & High level                                    & Low level                                   \\ \hline
\textbf{Application field}               & Computer networking, Energy efficient routing & Wireless Sensor Networks, Intrusion detection, Maximum network coverage \\ \hline
                                         &                                               &                                             \\ \hline
\end{tabular}
\end{table}

\subsection{Is there a general difference between high and low level inspired approaches?}
\textit{What is the difference? Is either one better? Are low level functions more advanced?}

The wanted result of bio-inspired approaches is to design a working solution based on only a simple set of rules \cite{dressler2010bio}. Hence it should not make a difference whether the chosen approach was inspired by higher or lower level functions of biological systems.

\textit{rubbish?}
Hylsberg et al describe in their article \cite{hylsberg2011bioinspired} how different biological systems and principles have been successfully applied in different contexts. Swarm intelligence, which includes Ant Colony Optimization, has usually been applied in routing in computer networks, optimal node deployment, node localization, and network clustering. Intercellular communication, which the writer of this paper believes to be the closest of the mentioned biological prinicples to Physarum, has been applied in coordination and control in distributed systems, network clustering, and security and privacy. Although intercellular communication seems like the best of the four categories (Swarm intelligence, Firefly synchronization, Artificial immune system, Intercellular communication) in the article, it is a bit forced since physarum polycephalum is a single celled organism and hence does not really communicate with other cells. ...?

\textit{
\begin{itemize}
	\item Why the explained bio-inspired approaches are used / why they are good.
	\item Some future aspects on possible directions of bio-inspired networking.
\end{itemize}
}

%============================================================

\bibliography{references}
\end{document}

