% For easier proof-reading, use the single-column, double-spaced layout:
\documentclass{IWORK2014}
% Final Paper use double-column, normal line spacing. Comment the
% above line and uncomment the following line when you are writing
% Full paper and Final paper!  
%\documentclass[cameraready]{IWORK2014}

\usepackage[hyphenbreaks]{breakurl}
\usepackage{hyperref}
\begin{document}

%=========================================================

\title{Bio-Inspired Networking}

\author{Juha Viljanen\\
        Aalto University School of Science \\
	\texttt{juha.o.viljanen@aalto.fi}}
\maketitle

%==========================================================

\begin{abstract}
\textit{Abstract will be here...}

\vspace{3mm}
\noindent KEYWORDS: bio-inspired, networking

\end{abstract}

%============================================================


\section{Introduction}

Nature provides different mechanisms to solve complex problems in pragmatic, efficient and elegant ways. There is a lot that we can learn from Nature and apply to the world of computing and networking. One can find analogies between the different players and functions of a specific system in nature and a technical problem for example in the networking domain \cite{dressler2010bio}.

Inspiration has classically been drawn from a higher level understanding of biological systems in Nature \cite{kroeker2011biology}, \cite{liu2012physarum}. A representative example of this is the sophisticated way ants passively communicate with each other while foraging. They do not need any centralized controlling unit nor do they ever need to have fysical contact with each other. By leaving and following the right feromone trails specified by a certain set of rules, they are able to to manage their foraging in an efficient and autonomous way. A similar solution has been adapted and applied to a modern routing protocol \cite{dressler2010bio}.

Today also solutions from low level biological systems, such as molecular systems \cite{kroeker2011biology} or simple cellular organisms \cite{liu2012physarum}, are adapted to the field of computer science. For example the neurologic development of fruit-flies has inspired a minimalistic and efficient algorithm for self-organizing distributed networks \cite{kroeker2011biology}.

\subsection{What is Bio-inpired networking?}
Bio-inpired approaches are employed in three main areas. These are computing, systems and networking. They are all technical solutions that draw their inspiration from Nature. In the area of computing bio-inspired approaches are exploited to improve the efficiency of computation algorithms for example in optimization processes or pattern matching. Regarding the systems area, research is ongoing to design system architectures of massivley distributed, collaborative systems. The area of networking is already benefiting from efficient and scalable networking solutions and autonomous organizing in a distributed environment. \cite{dressler2010bio}

\subsection{What makes Bio-inspired approaches attractive?} 
Bio-inspired approaches offer qualities attractive to the networking field. They offer feasable solutions for achieving the demanding charasteristics of next generation network architectures. The most challenging characteristics are the dynamic nature of mobile and ad-hoc networks and cognitive radio networks, autonomous operation in a infrastructureless network and communication in nano and micro scale networks \cite{dressler2010bio}. In large, heterogenous networks, the most efficient solutions do not usually include a centralized controlling entity, but favor a self-organizing, learning and evolving type of agents traversing and finding optimal routes through the network \cite{dressler2010bio}.

Biological systems often have a small set of simple rules that can be used to create complex behavior \cite{dressler2010bio} for solving challenging problems. These solutions usually include qualities such as self-organization \cite{kroeker2011biology}, adaptation to changing environmental conditions, fault tolerance, efficient management of scarce resources, collaboration and survival to harsh conditions \cite{dressler2010bio}.

\section{Background}
\textit{Definitions of commonly used terms here.}
\subsection{Wireless Sensor Network}
Wireless Sensor Networks consist of sensor nodes geographically distributed on an area \cite{nazi2013robust}. These sensor nodes collect data from their environment, process it by doing aggregation and filtering and forward it to other recievers in the network they are connected in. For WSNs it is important to be fault tolerant, since network failures may occurr for example due to one of its nodes running out of power. WSNs may be used in contexts of smart healthcare, disaster management, environment monitoring \cite{nazi2013robust} or even military applications \cite{liu2012physarum}.

\cite{kroeker2011biology}, \cite{liu2012physarum}, \cite{nazi2013robust}



\section{Results}

\textit{
\begin{itemize}
	\item List approaches here.
	\item How does their archetype in the biological system work?
	\item How is it applied?
	\item Explain the alogrithm.
\end{itemize}
}

\subsection{Bio-inspired solutions from high level functions of biologial systems}
\subsubsection{Ant Colony Optimization}
A classic application of Ant Colony Optimization (ACO) was the AntNet routing protocol that was able to exploit ACO making the protocol function more efficiently and without an outside controlling unit \cite{di1998antnet}.
\cite{kroeker2011biology}
More recently a new routing scheme called A-ESR has been proposed to make network elements and hence the whole Internet to be more energy efficient \cite{kim2012ant}, \cite{kim2011ant}.

\subsection{Bio-inspired solutions from low level functions of biologial systems}
\subsubsection{Neurologic development of fruit-flies}
\cite{dressler2010bio}

\subsubsection{Physarum Optimization}
The Physarum Optimization approach exploits the cellular computation model of the slime mold physarum polycephalum in wireless sensor networks (WSN) \cite{liu2012physarum}. It is used to solve the minimal exposure path problem of WSNs, which relateds to their worst case coverage. The approach is claimed to be simple and highly concurrent. It should also be useful for designing new graph-algorithms, routing protocols and Self-Organizing Network (SON) topologies.






\section{Conclusion}

\subsection{Is there a general difference between high and low level inspired approaches?}
\textit{What is the difference? Is either one better? Are low level functions more advanced?}

The wanted result of bio-inspired approaches is to design a working solution based on only a simple set of rules \cite{dressler2010bio}. Hence it should not make a difference whether the chosen approach was inspired by higher or lower level functions of biological systems.

\textit{
\begin{itemize}
	\item Why the explained bio-inspired approaches are used / why they are good.
	\item Some future aspects on possible directions of bio-inspired networking.
\end{itemize}
}

%============================================================

\bibliography{references}
\end{document}

