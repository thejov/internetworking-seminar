% For easier proof-reading, use the single-column, double-spaced layout:
\documentclass{IWORK2014}
% Final Paper use double-column, normal line spacing. Comment the
% above line and uncomment the following line when you are writing
% Full paper and Final paper!  
%\documentclass[cameraready]{IWORK2014}

\usepackage[hyphenbreaks]{breakurl}
\usepackage{hyperref}
\begin{document}

%=========================================================

\title{Bio-Inspired Networking}

\author{Juha Viljanen\\
        Aalto University School of Science \\
	\texttt{juha.o.viljanen@aalto.fi}}
\maketitle

%==========================================================

\begin{abstract}
  Abstract will be here...

\vspace{3mm}
\noindent KEYWORDS: bio-inspired, networking

\end{abstract}

%============================================================


\section{Introduction}

Nature has a way of solving complex problems in pragmatic, efficient and elegant ways. There is a lot that we can learn from Nature and apply to the world of computing and networking. One can find analogies between the different players and functions of a specific system in nature and a technical problem in for example the networking domain \cite{dressler2010bio}.

Inspiration has classically been drawn from a higher level understanding of biological systems in Nature \cite{kroeker2011biology}. A good example of this is the sophisticated way ants passively communicate with each other while foraging. They do not need any outside controlling unit nor do they ever need to have fysical contact with each other. By leaving and following the right feromone trails specified by a certain set of rules, they are able to to manage their foraging in an efficient and autonomous way. A similar solution has been adapted and applied to a modern routing protocol \cite{dressler2010bio}.

Today also solutions from low level biological systems, such as molecular systems, are adapted to the field of computer science \cite{kroeker2011biology}. For example the neurologic development of fruit-flies has inspired a minimalistic and efficient algorithm for self-organizing distributed networks \cite{kroeker2011biology}.

\subsection{What is Bio-inpired networking?}
Bio-inpired approaches are employed in three main areas. These are computing, systems and networking. They are all technical solutions that draw their inspiration from Nature. In the area of computing bio-inspired approaches are exploited to improve the efficiency of computation algorithms in for example optimization processes or pattern matching. Regarding the systems area, research is ongoing to design system architectures of massivley distributed, collaborative systems. The area of networking is already benefiting from efficient and scalable networking solutions and autonomous organizing in a distributed environment. \cite{dressler2010bio}

\subsection{What makes Bio-inspired approaches attractive?} 
Bio-inspired approaches offer qualities attractive to the networking field. They offer feasable solutions for achieving the demanding charasteristics of next generation network architectures. The most challenging characteristics are the dynamic nature of mobile and ad-hoc networks and cognitive radionetworks, autonomous operation in a infrastructureless network and communication in nano and micro scale networks \cite{dressler2010bio}. In large, heterogenous networks, the most efficient solutions do not usually include a centralized controlling entity, but favor a self-organizing, learning and evolving type of agents traversing and finding optimal routes through the network \cite{dressler2010bio}.

Biological systems often have a small set of simple rules that can be used to create complex behavior \cite{dressler2010bio} for solving challenging problems. These solutions usually include qualities such as self-organizing \cite{kroeker2011biology}, adapting to changing environmental conditions, fault tolerance, efficient management of scarce resources, collaborating and surviving harsh conditions \cite{dressler2010bio}.

\section{Background}
Definitions of commonly used terms here.

\section{Results}

\begin{itemize}
	\item List most commonly used approaches here.
	\item How does their archetype in the biological system work?
	\item How is it applied?
	\item Explain the alogrithm.
\end{itemize}

\section{Conclusion}
\begin{itemize}
	\item Why the explained bio-inspired approaches are used / why they are good.
	\item Some future aspects on possible directions of bio-inspired networking.
\end{itemize}


%============================================================

\bibliography{references}
\end{document}

